% !TEX root = Master.tex


Nowadays, online shopping is gradually becoming people's favourite purchasing standard. As designer and fashion brands adjust to this new way of shopping, they are not only promoting their products via third party 	providers, but also have their own e-Commerce websites. Likewise, \textit{adidas} has grown its eCom channel tremendously over the past few years and has gained a large pool of casual and regular customers.\\

The aim of some use cases of the adidas Advanced Analytics Hub is to generate sales\footnote{By sales we actually mean sale quantities in units throughout this thesis} forecasts for individual articles, usually on a weekly or monthly level. This is not always trivial since industrial big data are quite noisy, e.g. different types of campaigns and promotions influence the demand quantity dramatically over time. The purpose of this thesis is to capture a dependence structure between article demand quantities over time, applied to transactional eCom data. Primary modelling focus will be on particular clusters of article demand quantities and extensions on individual articles, will be outlined. \\

Another latent effect is sales cannibalization between newer and older articles or articles of similar traits. To identify this effect in a quantifiable way, there will be suggestions for further methodologies towards the end. \\

In the remaining of Chapter \ref{sec:introduction}, a brief overview of the sports brand adidas is given in Section \ref{ssec:adidas} and in Section \ref{ssec:data_sources} we will have a first look into our data sources and a data dictionary. Chapter \ref{sec:theory_and_methods} introduces some notions on relevant statistical methodology. In Chapter \ref{sec:copulas_and_dependence_structures}, we will have a closer look into the copula framework which comprises the major ingredient of the modelling part of this thesis. In Chapter \ref{sec:data_exploration} we perform some exploratory data analysis to deep dive into the behaviour of article sales and to step-wise investigate some hierarchical properties of our data.  Chapter \ref{sec:modelling} analyses based on some preliminary work the results of modelling copulas to the data and we will examine some diagnostics using different approaches. A final conclusion will be summarized in Chapter \ref{sec:conclusion}, where we point out the main findings of this thesis and further potential study directions.





% !TEX root = Master.tex


Nowadays, online shopping is gradually becoming people's favourite purchasing standard. As designer and fashion brands adjust to this new way of shopping, they are not only promoting their products via third party 	providers, but also have their own e-Commerce websites. Likewise, \textit{adidas} has grown its eCom channel tremendously over the past few years and has gained a large pool of casual and regular customers.
\\

The aim of some use cases of the adidas Advanced Analytics Hub is to generate sales\footnote{By sales we actually mean sale quantities in units throughout this thesis.} forecasts for individual articles, usually on a weekly or monthly level. This is not always trivial, since industrial big data are quite noisy, e.g. different types of campaigns and promotions influence the demand quantity dramatically over time. The purpose of this thesis is to capture dependence structures between article demand quantities over time, applied to transactional eCom data. Primary modelling focus lies on special clusters of article demand quantities, where we analyze time-varying correlation structures. Another latent effect is sales cannibalization between newer and older articles or articles of similar traits. To identify such causal effects in a quantifiable way, frameworks applied to individual articles are outlined afterwards serving as a suggestion for further investigation.
\\

For the remaining of Chapter \ref{sec:introduction}, a brief overview of the sports brand adidas is given in Section \ref{ssec:adidas} and in Section \ref{ssec:data_sources} we have a first look into our data sources and a data dictionary is introduced to get familiar with the data. Chapter \ref{sec:theory_and_methods} introduces some notions on statistical methodologies relevant for this thesis. In Chapter \ref{sec:copulas_and_dependence_structures}, we take a closer look into theoretical aspects of the copula framework, which comprises the major ingredient of this thesis' modelling part. In Chapter \ref{sec:data_exploration}, some exploratory data analysis is performed to delve into the patterns of article sales and to step-wise investigate some hierarchical properties of the data.  Chapter \ref{sec:modelling} analyses the results of modelling conditional copulas to clusters of the data based on some preliminary work on the marginal distributions via GAMLSS. Thereafter, some diagnostics are examined. Moreover, in Section \ref{ssec:article_dependencies} a sketch for inferring causal effects between demand quantities of individual articles using dynamic Bayesian networks is proposed for research beyond this thesis. A final conclusion is summarized in Chapter \ref{sec:conclusion}, where we point out the main findings of this thesis and further potential study directions.





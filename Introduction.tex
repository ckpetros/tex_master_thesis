% !TEX root = Master.tex


In today's age, online shopping is gradually becoming people's favourite purchasing standard. As designer and fashion brands adjust to this new style of shopping, they are not only promoting their products via third party 	providers, but also have their own e-Commerce websites. As such, \textit{adidas} has grown its eCom channel over the past few years tremendously and has gained a large pool of casual and regular customers.\\

The scope of some use cases of the adidas Advanced Analytics Hub is to generate sales\footnote{By sales we mean the unit sales} forecasts for individual articles, usually on a weekly or monthly level. This is not always trivial since industrial big data are quite noisy, e.g. different types of campaigns and promotions influence the demand quantity dramatically over time. Another latent effect is sales cannibalization between newer and older articles or articles of similar texture. To identify this effect in a quantifiable way, the purpose of this thesis is to capture a dependence structure between article demand quantities over time, applied to transactional eCom data. \\

In the remaining of this chaper, a brief overview of the sports brand adidas is given in Section \ref{ssec:adidas} and in Section \ref{ssec:data_sources} we will have a first look into our data sources and a data dictionary. Chapter \ref{sec:theory_and_methods} introduces some notions on relevant statistical methodology. In Chapter \ref{sec:copulas_and_dependence_structures}, we will have a closer look into the copula framework which comprises the major ingredient of the modelling part of this thesis. In Chapter \ref{sec:data_exploration} we perform some exploratory data analysis to deep dive into the behaviour of article sales and to step-wise investigate some hierarchical properties of our data.  Chapter \ref{sec:modelling} analyses the results of modelling copulas to the data and we will examine some diagnostics. A final conclusion will be summarised in Chapter \ref{sec:conclusion}, where we point out the main findings of this thesis.





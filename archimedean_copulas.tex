% !TEX root = Master.tex

Unlike implicit copulas, \textit{explicit copulas} can be specified directly by taking into account certain constructional principles. The most important aspects of a such explicit copulas, in particular \textit{archimedean copulas}, are showcased in this subsection. Archimedian copulas are of the general form
\begin{equation}C(\boldsymbol{u})=\phi^{-1}\left(\phi\left(u_{1}\right)+\cdots+\phi\left(u_{d}\right)\right)
\label{eq:archimedean_copula_1}
\end{equation}
or equivalently

where the function $\phi:[0,1] \rightarrow [0, \infty)$ is the \textit{(archimedean) generator} and satisfies the following properties:
\begin{itemize}
\item $\phi$ is strictly decreasing in the entire domain $[0, 1]$
\item We set $\phi (1) = 0$
\item If  $\phi(0)=\lim \limits _{u \rightarrow 0^{-}} \phi(u)= \infty$, then $\phi$ is called \textit{strict}.
\end{itemize}
Based on \autoref{eq:archimedean_copula_1} and according to the form of the generator, we can construct several copula families. Three of the most popular ones are the \textit{Gumbel}, the \textit{Clayton} and the \textit{Frank} \textit{copula}, which will be discussed. \footnote{We will look into these copulas for the bivariate case ($d=2$) only.} The advantage of such copulas lies in the fact that they interpolate between certain fundamental dependency structures.\\

\textbf{Clayton Copula}\\
If the generator takes on the form
\begin{equation}
\phi_{C l}(u)=\frac{1}{\theta}\left(u^{-\theta}-1\right)
\end{equation}
then we obtain the \textit{Clayton copula} given by
\begin{equation}
C_{\theta}^{C l}\left(u_{1}, u_{2}\right)=\left(\max \left\{u_{1}^{-\theta}+u_{2}^{-\theta}-1,0\right\}\right)^{-\frac{1}{\theta}},
\end{equation}
where $\theta \in[-1, \infty) \backslash\{0\}$.
For $\theta > 0$ the generator of the Clayton copula is strict and we arrive at 
\begin{equation}
C_{\theta}^{C l}\left(u_{1}, u_{2}\right)= (u_{1}^{-\theta}+u_{2}^{-\theta}-1)^{-\frac{1}{\theta}}.
\end{equation}

Note that for $\theta=-1$, we obtain the lower Fr\'echet-Hoeffding bound, whereas for the limits $\theta \rightarrow 0$ and $\theta \rightarrow \infty$ we arrive at the independence copula and the comonotonicity copula respectively.

\hfill $\square$ \\

\textbf{Gumbel Copula}\\
If the generator takes on the form
\begin{equation}
\phi_{G u}(u)=(-\ln u)^{\theta}, \quad \theta \in [1, \infty),
\end{equation}
then we arrive at the \textit{Gumbel copula} given by
\begin{equation}
C_{\theta}^{G u}\left(u_{1}, u_{2}\right)=\exp \left[-\left(\left(-\ln u_{1}\right)^{\theta}+\left(-\ln u_{2}\right)^{\theta}\right)^{\frac{1}{\theta}}\right].
\end{equation}
Note that for $\theta= 1$, we obtain the independence copula, while for $\theta \rightarrow \infty$ the Gumbel copula converges to the comonotonicity copula.

\hfill $\square$ \\


\textbf{Frank Copula}\\
If the generator takes on the form
\begin{equation}
\ln \left(e^{-\theta}-1\right)-\ln \left(e^{-\theta u}-1\right), \quad \theta \in \mathbb{R} \backslash\{0\},
\end{equation}
we obtain the \textit{Frank copula} given by
\begin{equation}
C_{\theta}^{F r}\left(u_{1}, u_{2}\right)=-\frac{1}{\theta} \ln \left(1+\frac{\left(e^{-\theta u_{1}}-1\right) \cdot\left(e^{-\theta u_{2}}-1\right)}{e^{-\theta}-1}\right)
\end{equation}

\hfill $\square$ \\


MAYBE MORE ON THESE WITH SOME PRETTY PLOTS






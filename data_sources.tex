% !TEX root = Master.tex

Throughout each season, transactional data were collected from online purchases of the sports brand's e-Commerce website. Specifically, weekly sales data for western European countries consisting of 109 observed weeks in total were provided. A short description is depicted in \autoref{tab:transactional_data}.\\

\begin{table}[H]
\setlength\arrayrulewidth{1pt}  
\centering
\begin{adjustbox}{max width=\textwidth}
\begin{tabular}{|l|l|l|}
\hline
\rowcolor{lightgray} 
\textbf{Column}         & \textbf{Description}                                                                                                        & \textbf{Values}              \\ \hline
week\_id                & Calender week of a specific year (YYYYWW)                                                                         & Factors:\textit{ 201648, ..., 201852}          \\ \hline
article\_number         & Unique article identification number (article ID)                                                                           & Factors: \textit{10669, 10, ...       }        \\ \hline
min\_date\_of\_week     & \begin{tabular}[c]{@{}l@{}} Minimum date of the respective week; always a Monday \\ (YYYY-MM-DD)                                                          \end{tabular} & Dates: \textit{2016-11-28, ..., 2018-12-24}  \\ \hline
art\_min\_price         & Minimal recorded price of the article                                                                                       & Non-negative (integer) value \\ \hline
month\_id               & Calender month of a specific year (YYYYMM)                                                                               & Factors: \textit{201612, ..., 201812}          \\ \hline
season                  & \begin{tabular}[c]{@{}l@{}}Season of year (format: SSYY) \\ (Spring-Summer [SS]:  December - May)\\  Fall-Winter [FW]: June - November)\end{tabular} & \begin{tabular}[c]{@{}l@{}}  Factors: \\ \textit{SS17, FW17,} \\\textit{ SS18, FW18, SS19} \end{tabular} \\

 \hline
bf\_w                   & Weekly "Black Friday" promotion intensity of the article                                                                    & Between \textit{0} and \textit{1}              \\ \hline
ff\_w                   & Weekly "Friends \& Family" promotion intensity of the article                                                               & Between \textit{0} and \textit{1}              \\ \hline
ot\_w                   & Weekly article promotion intensity of "Other" type                                                                          & Between \textit{0} and \textit{1}              \\ \hline
gross\_demand\_quantity & Weekly amount of added articles to shopping cart                                                                            & Non-negative (integer) value \\ \hline
base\_price\_locf       & Retail price of the article without any discounts                                                                           & Non-negative (integer) value \\ \hline
total\_markdown\_pct    &                                                                                                          Total markdown percentage of the article &    Non-negative                           \\ \hline
day\_of\_month          & Day of the month                                                                                                            & Integers: \textit{1 - 31}                       \\ \hline
month\_of\_year         & Month of the year                                                                                                           & Factors: \textit{January, ..., December}       \\ \hline
year                    & Year                                                                                                                        & Integers: \textit{2016, 2017, 2018}             \\ \hline
week\_of\_year          & Week of the year                                                                                                            & Integers: \textit{1 - 52}                       \\ \hline
\end{tabular}
\end{adjustbox}
\caption{Transactional raw data description from online purchases of western European countries}
\label{tab:transactional_data}
\end{table}

Due to legal regulations of the company, some columns had to undergo anonymization in order for the data to be released. To ensure data protection and confidentiality, numeric variables (with exception of time-indicating columns) were transformed. As a consequence for the analysis part, most integer values were converted to float numbers. This fact should be kept in mind by the reader, since the above table serves as a reminder and reference point for the data documentation.\\

Another peculiarity of this setup is to be considered too. We often refer to the variable \textit{gross demand quantity} as \textit{sales}, even though it is obviously not exactly the same. In the e-Commerce environment, there are several stages before the purchase is complete, e.g. addition to cart, removal from cart, proceeding to checkout \& even the return of bought articles. Targeting the articles added to cart, i.e. the (gross) demand quantity, provides the optimal data extraction for analytical purposes and is the closest to adequately model the dependence structure between net sales of articles.\footnote{Gross demand quantity is our target as we follow the adidas norm.}  \\

Besides the transactional data, attributes of the articles are provided and described in \autoref{tab:article_master_data}. Some attributes of special importance are explained in more detail later on in Chapter \ref{sec:data_exploration}. \\

\begin{table}[H]
\setlength\arrayrulewidth{1pt}  
\centering
\begin{adjustbox}{max width=\textwidth}
\begin{tabular}{|l|l|l|}
\hline
\rowcolor{lightgray}
\textbf{Column}           & \textbf{Description}                                   & \textbf{Values (all Factors)}                 \\ \hline
article\_number           & Unique article identification number (article ID)      & \textit{10669, 10, ...}                                \\ \hline
gender                    & Gender type of the article (Men, Women, Unisex)        & \textit{M, W, U}                                       \\ \hline
age\_group                & Age group of the article (Adult, Infant, Junior, Kids) & \textit{A, I, J, K}                                    \\ \hline
key\_category\_descr      & Key category of the article & \textit{KC\_1, ..., KC\_15}   \\ \hline
key\_category\_cluster\_descr      & Key category cluster of the article & \textit{KCC\_1, ..., KCC\_9}   \\ \hline
product\_division\_descr  & Product division of the article                        & \textit{Apparel, Footwear, Hardware}                   \\ \hline
product\_group\_descr     & Product group of the article                           & \textit{Bags, Balls, Footwear Accessories, Shoes, ...} \\ \hline
color                     & Consolidated color group of the article                & \textit{Beige, Black, Brown, Orange, Pink, Red, ... }  \\ \hline
sports\_category\_descr   & Sports category of the article                         & encoded: \textit{SC\_1, ..., SC\_22 }                  \\ \hline
sales\_line\_descr        & Sales line of the article                              & encoded: \textit{SL\_1, ..., SL\_379 }                 \\ \hline
business\_unit\_descr     & The article's Business Unit membership                 & encoded:\textit{ BU\_1, ..., BU\_18   }                \\ \hline
business\_segment\_descr  & The article's Business Segment membership              & encoded: \textit{BS\_1, ..., BS\_49  }                 \\ \hline
sub\_brand\_descr         & Sub-brand of the article                               & encoded: \textit{sub-brand\_1, ..., sub-brand\_4 }     \\ \hline
item\_type                & Item type of the article                               & encoded: \textit{IT\_1, ..., IT\_171 }                 \\ \hline
brand\_element            & Brand element of the article                           & encoded: \textit{BE\_1, ..., BE\_131 }                   \\ \hline
product\_franchise\_descr & Product franchise of the article                       & encoded: \textit{franchise\_1, ..., franchise\_72}     \\ \hline
product\_line\_descr      & Product line of the article                            & encoded: \textit{PL\_1, ..., PL\_105 }                 \\ \hline
franchise\_bin            & Franchise indicator of the article                     &\textit{Franchise, Non-Franchise }                     \\ \hline
category                  & Category of the article                                & encoded: \textit{category\_1, category\_2}             \\ \hline
\end{tabular}
\end{adjustbox}
\caption{Article attribute data}
\label{tab:article_master_data}
\end{table}

%Plenty of additional information is stored in the database, but we are %neglecting columns omitted in these tables, as they are redundant, %already summarized, transformed or simply do not provide any value.\\
Overall, these are the primary data sources and we are working with data collected over two years, namely the years 2017 and 2018, while some transactions of late 2016 are attached marginally. In summary, after joining the transactional observations to the article attributes by the article ID, this translates to a dataset of 587,127 instances including 26,203 distinct articles and over 30 variables. 


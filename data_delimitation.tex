% !TEX root = Master.tex

On the lowest product hierarchy level, the number of unique articles is quite large for models to handle. In \autoref{tab:articles_per_kcc}, we can see that our key category clusters of interest, 2, 6 and 8, have 1587, 9746 and 2012 articles respectively. Although we only work on those \acp{KCC}, even advanced techniques like vine copulas are struggling to handle such dimensions. \\

\begin{table}[H]
\setlength\arrayrulewidth{1pt}  
\centering
\begin{adjustbox}{max width=\textwidth}\
\begin{tabular}{|
>{\columncolor{lightgray}}c |c|c|c|c|c|c|c|c|c|}
\hline
\textbf{Key Category Cluster} & 1  & 2    & 3    & 4   & 5    & 6    & 7   & 8    & 9    \\ \hline
\textbf{Unique Articles}      & 14 & 1,587 & 4,151 & 278 & 1,713 & 9,746 & 374 & 2,012 & 6,328 \\ \hline
\end{tabular}
\end{adjustbox}
\caption{Articles per Key Category Cluster}
\label{tab:articles_per_kcc}
\end{table}

We have to take into consideration different models for each group. On top, further data delimitation shall reduce the dimensionality even further. Referring back to \autoref{tab:transactional_data}, we know that our last week of observation is "2018-12-24". We will restrict ourselves to articles that were not taken off the assortment before that date because they will most probably not be put back for sale.\footnote{We neglect the possibility of an item to return in the future.} After setting these boundaries, we arrive at a considerably smaller magnitude shown in \autoref{tab:articles_per_kcc_after_max_week}. \\



\begin{table}[H]
\setlength\arrayrulewidth{1pt}  
\centering
\begin{adjustbox}{max width=\textwidth}\
\begin{tabular}{|
>{\columncolor{lightgray}}c |c|c|c|}
\hline
\textbf{Key Category Cluster} & 2  & 6    & 8     \\ \hline
\textbf{Unique Articles}      & 39 & 256 & 58  \\ \hline
\end{tabular}
\end{adjustbox}
\caption{Articles per Key Category Cluster after data delimitation}
\label{tab:articles_per_kcc_after_max_week}
\end{table}


CARRY OVER IF WE GO AWAY FROM IN-SAMPLE







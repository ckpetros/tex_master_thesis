% !TEX root = Master.tex

In spite of the nuisances that eCom data exhibit, parametric approaches are still feasible in order to find patterns in the data and fit appropriate models. As seen in the three above subsections (\ref{sssec:kcc_26}, \ref{sssec:kcc_28} and \ref{sssec:kcc_68}), for each relevant key category cluster we are able to gain a good understanding of the underlying data generating process of the demand quantities. A logarithmic scale on the demand quantities ensures that extreme values are compressed, such that suitable distributions can be applied. The log-scaled quantities are treated as responses belonging to the ex-Gaussian distribution family within the context of \ac{GAMLSS}, where flexible distribution parameters can be estimated. Especially analyses of the residuals underline the positive outcomes. In the following section, we take advantage of those residuals as they practically approximate normally distributed values to pursuit estimation of temporal dependence structures.
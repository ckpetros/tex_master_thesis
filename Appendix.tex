% !TEX root = Master.tex


The summary outputs for the \ac{GAMLSS} fits of the log-scaled demand quantities are printed below in R outputs \ref{output:gamlss_fit_kcc_2_try20},  \ref{output:gamlss_fit_kcc_6_try1} and \ref{output:gamlss_fit_kcc_8_try1}.
\\


\inputRoutput[caption={GAMLSS Fit on KCC 2}, numbers=left,numberstyle=\tiny, label=output:gamlss_fit_kcc_2_try20]{gamlss_fit_kcc_2_try20.txt}




\inputRoutput[caption={GAMLSS Fit on KCC 6}, numbers=left,numberstyle=\tiny, label=output:gamlss_fit_kcc_6_try1]
{gamlss_fit_kcc_6_try1.txt}




\inputRoutput[caption={GAMLSS Fit on KCC 8}, numbers=left,numberstyle=\tiny, label=output:gamlss_fit_kcc_8_try1]
{gamlss_fit_kcc_8_try1.txt}




The summaries of \ac{GJRM} fits on pairs of the residuals from the previous step are printed in R outputs \ref{output:summary_kcc_26}, \ref{output:summary_kcc_28} and \ref{output:summary_kcc_68} below.
\\


\inputRoutput[caption={Summary of GJRM fit on key category clusters 2 \& 6},numbers=left,numberstyle=\tiny, label=output:summary_kcc_26]{summary_kcc_26.txt}



\inputRoutput[caption={Summary of GJRM fit on key category clusters 2 \& 8},numbers=left,numberstyle=\tiny, label=output:summary_kcc_28]{summary_kcc_28.txt}



\inputRoutput[caption={Summary of GJRM fit on key category clusters 6 \& 8},numbers=left,numberstyle=\tiny, label=output:summary_kcc_68]{summary_kcc_68.txt}






When the three sets of residuals are passed into the \textit{fitDist()} function of the \textit{gamlss} package, it returns multiple suggestions of parametric distribution that can be used to fit to the data, ordered by ascending \ac{AIC}. An excerpt is given in R output \ref{output:gamlss_distributions_aic}. 
\\

\inputRoutput[caption={AIC values of distribution fits on KCC with \textit{fitDist()} function of the \textit{gamlss} package},numbers=left,numberstyle=\tiny, label=output:gamlss_distributions_aic]{gamlss_distributions_aic.txt}


\textbf{Selection of copula for the pair \ac{KCC} 2 \& \ac{KCC} 6} \\ 
With the help of the function \textit{BiCopSelect()} of the R package \textit{VineCopula} \citep{nagler2019vinecopula}, we can select an appropriate bivariate copula family for the data. A set of possible copula families is considered and maximum likelihood estimations for each are carried out. Selection of a bivariate copula is based on the lowest \ac{AIC}. The normal and the t-copula are the first in the family set.
\\




\textbf{Exercise of dynamic Bayesian network on articles from KCC 1} \\ 
The below output is a summary from the \textit{dbnR\:\:learn\_dbn\_struc()} function call.
\\

\inputRoutput[caption={Learned structure of a DBN from articles in KCC 1},numbers=left,numberstyle=\tiny, label=output:dbnR_output_kcc_1]{dbnR_output_kcc_1.txt}





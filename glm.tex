% !TEX root = Master.tex
\textit{\acp{GLM}} are an extension of the classical \textit{\ac{LM}}
\begin{equation}
\begin{aligned}
&y_{i}=\beta_{0}+\beta_{1} x_{i 1}+\ldots+\beta_{k} x_{i k}+\varepsilon_{i}, &i=1, \ldots, n
\end{aligned}
\end{equation}
which in matrix notation can be written as
\begin{equation}
 \bm{y} = \bm{X}\bm{\beta} + \bm{\varepsilon} 
\end{equation}
where the response variable $y_i$ can take values from several probability distributions (e.g. Poisson, Binomial, Gamma, ...), which are members of the exponential family \citep{fahrmeir2003regression}. The linear predictor 
\begin{equation} 
\eta_i = \beta_{0}+\beta_{1} x_{i 1}+\ldots+\beta_{k} x_{i k}+\varepsilon_{i} = \bm{x'}_i \bm{\beta} + \bm{\varepsilon}
\label{eq:linear_predictor_glm}
\end{equation}
is passed through a \textit{response function h} (a one-to-one, twice differentiable transformation), such that
\begin{equation}
 E(y_i) = h(\eta_i),
\label{eq:response_function}
\end{equation}
where $h$ ensures that the expected value of the response variable belongs to the appropriate value range. The inverse of the response function, i.e.
\begin{equation}
g = h^{-1},
\label{eq:link_function}
\end{equation} 
is called the \textit{link function} and transforms the mean of the response's distribution to an unbounded continuous scale.

% MAXIMUM LIKELIHOOD ESTIMATION HERE??...




% !TEX root = Master.tex

In a first step, summary statistics of the inventory data are presented to get an insight of the forest and
correlations (see Section \ref{Overview variables}).
In a second step, a regression model is constructed to estimate the diameter of trees based on the height and
crown area from the inventory data. Additionally, the effect of tree species is examined. Subsequently, the
model is used to predict the diameter of the trees in the LiDAR dataset.

\renewcommand{\labelenumi}{\arabic{enumi}.}
\begin{enumerate}

\item \textbf{The Log-normal Regression Model} \\
The log-transformation (and back-transformation) is an established method introduced in 1941 by Finney (see [4]).
The transformation of the response yields in a log-normal model 
\begin{align*}
ln(y)  =  X\beta
\end{align*}
 where $X$ is the design matrix. The
random variable 
\begin{align*}
z = ln(y)
\end{align*}
 is normally distributed with $\mu_z$ and $\sigma_z^2$. 
The back-transformed random variable 
\begin{align*}
y = e^z
\end{align*}
 is log-normal
distributed. It can be shown that 
\begin{align*}
\mu_y = e^{x_i' \beta+\sigma_z^2/2}.
\end{align*}
$\sigma_z^2$ is estimated by the MSE $s_z^2$ and $e^{\sigma_z^2}$ can be considered as a correction term, which
due to its positivity always increases the back-transformation.

\item \textbf{Generalized Linear Models} \\
Instead of transforming the response itself, Generalized Linear Models (GLM) are used to instead transform the
mean. Thus, $\mu_i$ is connected to the linear predictor
\begin{align*}
\eta_i = x'_i\beta = \beta_0 + \beta_1 x_{i1} + \beta_2 x_{i2} + ... + \beta_k x_{ik}
\end{align*}
 through 
\begin{align*}
\mu_i = h(\eta_i) = h(x'_i \beta) \quad \text{or} \quad \eta_i = g(\mu_i),
\end{align*} 
  where $h$ is the response function and $g$ the link
function [5]. The Gaussian and gamma from the exponential family are chosen. The log function
as link is chosen for the same reason as in the log-normal model. The transformation of the mean can result in
substantially different result for the Gaussian model with log link compared to the log-normal model.
The dispersion parameter 	 (see Table \ref{tab:Prediction Models}) of the Gaussian model is just the $\sigma^2$ and $v^{-1}$ (inverse scale) for the gamma
model [5].

\end{enumerate}


Bias correction is performed on compartment level, where inventory data is sparse. K-means clustering will group
compartments with similar tree structure allowing for a richer inventory set used for bias correction.

\textbf{K-means clustering} \\
K-means clustering is a relatively simple iterative clustering procedure. In short, k centroids are set randomly into the data space.


\begin{enumerate}
\renewcommand{\labelenumi}{\arabic{enumi}}
\item For each data point (compartment), the distance to the k centroids is calculated.
\item Each data point is assigned to the centroid with its minimal distance forming k-groups.
\item The mean of all data points within the group is calculated, which is essentially the update of the centroid.

\end{enumerate}


This is repeated until there is no more change of the attribution of the compartments to the clusters (convergence is
reached). As the initiation of the k-centroids is random, results can vary. The number of k-clusters must be
defined prior to the start. Different numbers of cluster k should be tested, and decision can be made by
visualization of the clusters and/or change in the total within sum squares used to pick the best (heuristic) k.
Ultimately, the clusters should make sense in a way that experts can describe them. For more details see ref. [7].\\


For each of the resulting clusters of compartnents, parametric distributions are fitted on the values of the measured and predicted tree diameter. This leads to a posession of parameters of the (chosen kind of) distribution, such that tuning of such parameters allows for bias reduction.\\

\textbf{Parametric Distribution Fitting} \\
LiDaR as well as inventory diameters on a cluster level are subject to a fit of a preselected distribution. The chosen distribution needs to be reasonable in the sense that parameter or variable restrictions are well defined and the preselected distribution should properly reflect the observed distribution [13].\\

There are several methods for fitting a parametric distribution to the data. \\
In this case, Maximum Likelihood Estimation (MLE) is applied to estimate the parameters of the distribution, i.e. the Log-Likelihood function of the respective distribution is the objective of a maximization problem with respect to the distributional parameters [15].\\

\textbf{Gamma Distribution} \\
Let X be a random variable which follows a gamma distribution: $X \sim Ga(\alpha, \beta)$, where $\alpha$ is called the shape parameter and $\beta$ is called the rate parameter.\\
The corresponding probability density function is
\begin{align*}
f(x ; \alpha , \beta) = \dfrac{\beta^\alpha x^{\alpha -1}e^{- \beta x}}{\Gamma(\alpha)}
\quad \text{for} \quad x > 0 \quad \text{and} \quad \alpha , \beta > 0,
\end{align*}
where $\Gamma(\alpha)$ is the gamma function [14].\\
\textit{"The gamma distribution is not only a good model for waiting
times, but one for many nonnegative random variables of the continuous
type. For illustrations, the distribution of certain incomes could be modelled
satisfactorily by the gamma distribution, since the two parameters $\alpha$ and $\beta$
provide a great deal of flexibility."} [14]\\

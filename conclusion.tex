% !TEX root = Master.tex

The objective of this thesis was to identify interdependent relationships between product sales based on adidas eCom transactions. In order to spot such dependencies, we followed versatile strategies on different data formats  and concluded that different techniques resulted in different concepts of dependence, like correlation structures or causal connections.
\\

After getting acquainted with the weekly summarized data, which have been extracted from online transaction and grouped by the article ID's, we thoroughly deep dived into the patterns appearing in the data mainly by visual and analytical exploration. Despite thousands of articles being involved, we managed to divide the data exploration into three main parts; the global demand of the brand's products, the key category clusters of interest as well as the individual articles, where a sample was drawn to emphasize some obstacles linked to restrictions in the data. The key findings of this data exploration were related to promotion activities and markdowns. Especially Black Friday periods and elevated markdowns turned out to be driving sales in spite of the excessively persisting noise.
\\

Beginning with the key category clusters, we engaged in the modelling part by taking a two-step approach. The reason being that a thorough analysis on the marginal distributions , i.e. the (log-scaled) unit sales for each of the three targeted clusters, was desired before moving forward with the dependence structures. However, a distribution family with readily interpretable parameters (including a necessary skewness parameter) is not yet available in the \textit{GJRM} package. \\
Thus, the first step of this task was to implement \ac{GAMLSS} models to the clusters' log-sales using ex-Gaussian families for the responses. We additionally included covariates that stood out during earlier analyses and would likely induce significant effects on the parameters. Some interesting but peculiar contrasts between markdowns and promotion activities arised, leaving data quality topics open for discussion.\\
The residuals from the models in the first step were resorted to act as normally distributed responses in a second step, were the temporal correlation structures between pairs of clusters were determined via the \ac{GJRM} framework. By and large, adequateness of the model fits were backed by the diagnostics based on quantile residuals. Nonetheless, due to the data quality points addressed throughout the paper, the results shall not be taken for granted but rather serve as reference indicators for business users to monitor and tackle opposite directions in sales.
\\

As for the individual articles, other kind of questions regarding dependence structures are of more interest. They are predominantly related to cannibalization effects between product sales. Cannibalization effects are counterproductive, because a prerequisite for a healthy business is to not hurt itself by hindering their own products' sales. Thus, such effects ought to be prevented as much as possible.\\
 While exact quantification of inflow and outflow of sales between multiple articles needs more sophisticated approaches and is typically hard to figure out, determining causal effects, for instance within the context of (dynamic) Bayesian networks, is quite possible. \\
The paper is finalized with a superficial example of dynamic Bayesian networks applied to key category cluster 1, showcasing the advantages that \ac{DBN} algorithms are able to exploit. This thesis shall set the foundations for aiming towards such potential research directions in the future.






















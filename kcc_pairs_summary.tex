% !TEX root = Master.tex


In Section \ref{ssec:kcc_copulas}, we extend the modelling part of the previous section by employing bivariate copula additive models for all three pairs of key category clusters. We take advantage of the \ac{GJRM} framework by \cite{marra1605bivariate}. As response variables we use the residuals obtained from Section \ref{ssec:kcc_margins}, which are approximately normally distributed. Based on residual analyses, the models seem appropriately fit. Thus, dependence structures are estimated in a temporal manner, where we can compare whether demand quantities of cluster pairs move towards the same or the opposite direction including their correlation strengths. From a brand perspective, it is crucial to identify movements in the opposite direction and act accordingly. Since time-varying correlation structures are hard to validate, results should always be treated cautiously. In the next section, we are concerned with dependence structures of individual articles within the context of causality.
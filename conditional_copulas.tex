% !TEX root = Master.tex

Modelling of the marginal response distributions along with their dependence structure has been studied so far in a strictly parametric context, not considering any potentially available covariate information. In this section, we will broaden the copula framework by adding conditions given possible covariates for all model parameters, i.e. both for the parameters of the marginals as well as the copula parameter. All involved model parameters will receive \textit{structured additive predictors} (see Section \ref{ssec:gam}) to account for possible non-linear or random effects. We will summarily explore \textit{Structured Additive Conditional Copulas} and for extensive literature, good references to view are \cite{klein2016simultaneous}, \cite{vatter2019gamcopula} and \cite{marra1605bivariate}. \\

To get started, we define $(Y_1, Y_2)'$ to be independent bivariate responses and $\bm{\nu}$ being the information contained in covariates. Ergo, \autoref{eq:sklar} of Sklar's theorem can be extended to the conditional case
\begin{equation}
F_{1,2}\left(Y_{1}, Y_{2} | \bm{\nu} \right)=C\left(F_{1}\left(Y_{1} | \bm{\nu} \right), F_{2}\left(Y_{2} | \bm{\nu} \right) | \bm{\nu} \right)
\label{eq:sklar_conditional}
\end{equation}
in conjunction with all facets of Section \ref{ssec:intro_to_copulas} \citep{patton2006modelling}.\\
The marginal \acp{CDF} $F_{d}\left(y_{i d} | \bm{\nu}_i\right)$ for observations $i = 1,\ldots, n$ can also be stated as
\begin{equation}
F_{d}\left(y_{i d} | \vartheta_{i 1}^{(d)}, \ldots, \vartheta_{i K_{d}}^{(d)}\right), \quad d = 1, 2,
\end{equation}
i.e. the distribution $F_d$ has a total of $K_d$ parameters, denoted as $\vartheta_{i 1}^{(d)}, \ldots, \vartheta_{i K_{d}}^{(d)}$.
To relate all parameters of the marginals to structured additive predictors $\eta_i^{\vartheta_k^{(d)}},  k = 1,\ldots, K_d$ consisting of the covariates $\bm{\nu}_i$ (see Section \ref{ssec:gam}), we employ strictly increasing response mappings $h_k^{(d)}$ to ensure proper domain allocation, i.e.
\begin{equation}
\vartheta_{i k}^{(d)}=h_{k}^{(d)}(\eta_{i}^{\vartheta_{k}^{(d)}}).
\label{eq:parameter_mapping}
\end{equation}
\\

Assuming that the parameters of the copula can also depend on covariates $\bm{\nu}_i$ while Sklar's theorem applies as usual, the left-hand side of \autoref{eq:sklar_conditional} can equivalently be stated as  
$$
F_{1,2}(y_{i 1}, y_{i 2} | v_{i})= F_{1,2}(y_{i 1}, y_{i 2} | \vartheta_{i 1}^{(1)}, \ldots, \vartheta_{i K_{1}}^{(1)}, \vartheta_{i 1}^{(2)}, \ldots,
\vartheta_{i K_{2}}^{(2)}, \vartheta_{i 1}^{(c)}, \ldots, \vartheta_{i K_{c}}^{(c)}),
$$
where the last share of parameters $\vartheta_{i 1}^{(c)}, \ldots, \vartheta_{i K_{c}}^{(c)}$ belong to the copula. Similar to \autoref{eq:parameter_mapping}, the copula parameters are modelled as $\vartheta_{i k}^{(c)}=h_{k}^{(c)}(\eta_{i}^{\vartheta_{k}^{(c)}})$ with $K_c$ being the number of parameters.














% !TEX root = Master.tex


%\vspace{2cm}


While transactional data storage of the adidas eCom website is constantly increasing in size, analytical products generating data-driven insights became a vital part of the business. For this thesis, two years of weekly data were provided in order to discover dependence structures between demand quantities of sports and fashion articles.
\\

Since eCom sales data are fairly sparse considering that there are tens of thousands of articles available online, targeting directly the articles is quite a challenge. Therefore, the focus firstly lies in modelling dependence structures of summarized groups of articles called "key category clusters", as they represent aggregated sport/fashion categories at adidas. Demand quantities of three specific clusters undergo thorough analysis along with critical features such as promotion activities and markdowns. To model temporal dependencies between the clusters, a two-step approach is carried out where we first have the demand quantities of the marginal responses fitted flexibly with respect to the mentioned features over time, using GAMLSS with exponentially modified Gaussian families. Secondly, the quantile residuals from the first step were used as normally distributed responses to obtain the pairwise dependence structures of the clusters with the help of Bivariate Copula Additive Models for Location, Scale and Shape. Both steps were carefully realized; dependence structures in form of time-varying correlation coefficients over 109 weeks were discovered and validity of the results were reviewed.
\\

As for the individual articles, some central data sufficiency issues were pointed out, highlighting the sparsity of the data. An exercise of dynamic Bayesian networks, which have the potential to detect causal effects between entities over time, was applied to distinct articles in one of the available key category clusters to demonstrate some of the advantages of these models against the data limitations. The thesis wraps up proposing such frameworks for future research.


\vspace{1cm}

\textit{\textbf{Keywords:} GAMLSS - Structured Additive Conditional Copula Regression - Generalized Joint Regression Models - Dynamic Bayesian Networks}



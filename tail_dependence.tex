% !TEX root = Master.tex

\textit{Coefficients of tail dependence} express the strength of the dependence in the extremes of distributions, i.e. the joint tails. We distinguish between \textit{lower} and \textit{upper tail dependence} between $X_j \sim F_j, j = 1,2$ and provided that the below limits exist, they are given by
\begin{equation}
\lambda_{l}=\lim \limits _ {q \rightarrow 0^+} P \left(X_{2} \leq F_{2}^{\leftarrow}(q) | X_{1} \leq F_{1}^{\leftarrow}(q)\right) 
\label{eq:lower_tail_dependence}
\end{equation}
and 
\begin{equation}
\lambda_{u}=\lim \limits _ {q \rightarrow 1^-} P \left(X_{2} > F_{2}^{\leftarrow}(q) | X_{1} > F_{1}^{\leftarrow}(q)\right).
\label{eq:upper_tail_dependence}
\end{equation}
If $\lambda_l$ (or $\lambda_u$) $=0$, then we say that $X_1$ and $X_2$ are \textit{asymptotically independent} in the lower (or upper) tail,\footnote{Not necessarily true for the other way around} otherwise we have lower (or upper) tail dependence.\\
For continuous \acp{CDF} and by using Bayes' theorem, these expressions can be re-written to
$$
\begin{aligned}
\lambda_{l} &=\lim _{q \rightarrow 0^+} \frac{P\left(X_{2} \leq F_{2}^{\leftarrow}(q), X_{1} \leq F_{1}^{\leftarrow}(q)\right)}{P\left(X_{1} \leq F_{1}^{*}(q)\right)} \\
&=\lim _{q \rightarrow 0^+} \frac{C(q, q)}{q}
\end{aligned}
$$
and similarly
$$
\lambda_u = 2-\lim _{q \rightarrow 1^-} \frac{1-C(q, q)}{1-q}.
$$
Therefore, tail dependencies can be assessed by means of the copula itself when approaching the points $(0,0)$ and $(1,1)$. In addition, for all radially symmetric copulas (e.g. the bivariate Gaussian or the t-copula) we have $\lambda_l = \lambda_u = \lambda$.\\
Some examples are:
\begin{itemize}
\item Clayton: $\lambda_l = 2^{-1/ \theta}$, $\lambda_u = 0$ (only lower tail dependence, see \autoref{fig:clayton_plots})
\item Gumbel: $\lambda_l = 0$, $\lambda_u = 2 - 2^{1/ \theta}$ (only upper tail dependence, see \autoref{fig:gumbel_plots})
\item Frank: $\lambda_l = 0$, $\lambda_u = 0$ (no tail dependence, see \autoref{fig:frank_plots})
\end{itemize}
Following such guidelines, the choice of a practicable copula can be facilitated. \autoref{tab:copula_relationships} displays the relationships between dependence measures of various copulas.\\



\begin{table}[H]
\setlength\arrayrulewidth{1pt}  
\centering
\begin{adjustbox}{max width=\textwidth}\
\begin{tabular}{|
>{\columncolor[HTML]{C0C0C0}}c |c|c|c|c|}
\hline
\backslashbox{\textbf{Copula}}{\textbf{Measure}} & \cellcolor{lightgray}\textbf{$\bm{\tau}$}          & \cellcolor{lightgray}\textbf{$\bm{\rho_s}$}                     & \cellcolor{lightgray}\textbf{$\bm{\lambda_l}$}                              & \cellcolor{lightgray}\textbf{$\bm{\lambda_u}$}                              \\ \hline
\textbf{Gaussian}       & $\frac{2}{\pi}arcsin(\rho)$                      & $\frac{6}{\pi}arcsin(\rho)$                                   & 0                                                                         & 0                                                                         \\ \hline
\textbf{Student's t}    & $\frac{2}{\pi}arcsin(\rho)$                      & -                                                             & \multicolumn{1}{l|}{$2 T_{\nu+1}(\sqrt{\frac{(\nu+1)(1-\rho)}{1+\rho}})$} & \multicolumn{1}{l|}{$2 T_{\nu+1}(\sqrt{\frac{(\nu+1)(1-\rho)}{1+\rho}})$} \\ \hline
\textbf{Clayton}        & $\frac{\theta}{\theta + 2}$                      & -                                                             & $2^{-1 / \theta}$                                                         & 0                                                                         \\ \hline
\textbf{Gumbel}         & $\frac{\theta - 1}{\theta}$                      & -                                                             & 0                                                                         & $2-2^{1 / \theta}$                                                        \\ \hline
\textbf{Frank}          & $1-\frac{4}{\theta}\left(4-D_{1}(\theta)\right)$ & $1-\frac{12}{\theta}\left(D_{1}(\theta)-D_{2}(\theta)\right)$ & 0                                                                         & 0                                                                         \\ \hline
\end{tabular}
\end{adjustbox}
\caption{Bivariate relationships in copula families, with $T_{\nu}$ being the Student's t-distribution function with $\nu$ degrees of freedom and $D_{k}(x)=\frac{k}{x^{k}} \int_{0}^{x} \frac{t^{k}}{e^{t}-1} d t$ being the Debye function \citep{stanfordphd}}
\label{tab:copula_relationships}
\end{table}





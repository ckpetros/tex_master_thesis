% !TEX root = Master.tex

\textit{Additive Models} expand models with just a linear predictor  
$$
\eta_{i}^{lin} = \beta_{0}+\beta_{1} x_{i1}+\ldots+\beta_{k} x_{i k}
$$
(such as the \ac{LM}) to 
\begin{equation}
y_{i} = \eta_{i}^{add} + \varepsilon_{i} ,
\label{eq:additive_model}
\end{equation}
where 
\begin{equation}
\eta_{i}^{a d d}=f_{1}\left(z_{i 1}\right)+\ldots+f_{q}\left(z_{i q}\right)+\eta_{i}^{l i n}, \quad i = 1, \ldots, n.
\end{equation}

The functions $f_{1}(z_{1}), \ldots, f_{q}(z_{q})$ are non-linear univariate \textit{smooth effects} of the \textit{continuous} covariates $z_1, \ldots, z_q$ and are defined as
\begin{equation}
f_{j}\left(z_{j}\right)=\sum_{l=1}^{d_{j}} \gamma_{j l} B_{l}\left(z_{j}\right)
\end{equation}
with $B_{l}\left(z_{j}\right)$ being \textit{basis functions} for $j = 1, \ldots, q$ and $d_j$ the number of basis functions for covariate $z_j$. The regression coefficients of the basis functions $B_l(z_j)$ are labeled as $\gamma_{jl}$. There is a wide variety of basis functions which can be used to flexibly model the data in a non-parametric manner. For more content on basis functions the reader can refer to \cite{wood2017generalized} and \cite{fahrmeir2003regression}. The basis functions evaluated at the observed covariate values are summarized in the design matrices $\bm{Z}_1, \ldots, \bm{Z}_q$ and the additive model \ref{eq:additive_model} can be written in matrix notation as
\begin{equation}
\bm{y} = \bm{Z}_1 \bm{\gamma}_1 + \ldots + \bm{Z}_q \bm{\gamma}_q + \bm{X} \bm{\beta} + \bm{\varepsilon}.
\label{eq:gam_matrix_notation}
\end{equation}
Accordingly, the vector of function values evaluated at the observed covariate values $z_{1j}, \ldots, z_{nj}$ is denoted by $\bm{f}_j = (f_j(z_{1j}), \ldots, f_j(z_{nj}))' $ and therefore $\bm{f}_j = \bm{Z}_j \bm{\gamma}_j$. To ensure identifiability of the additive model, the smooth functions $f_j(z_j)$ are centered around zero, such that
$$
\sum_{i=1}^{n} f_{1}\left(z_{i 1}\right)=\ldots=\sum_{i=1}^{n} f_{q}\left(z_{i q}\right)=0.
$$
\\

A convenient trait of additive models is that they also support the incorporation of random effects. Random coefficient terms can straightforwardly be added to the model.
%Analogously to Section \ref{ssec:mixed_models}, 
Data are considered to be measured in a longitudinal setting with individuals  $i=1, \ldots, m$ observed at times $t_{i1} < \ldots < t_{ij} < \ldots < t_{i_{n_i}}$ or clustered data with subjects $j=1, \ldots, n_i$ in clusters $i=1, \ldots, m$. \Ac{wlog},\footnote{for the indexes} we can simply add to \autoref{eq:gam_matrix_notation} the terms $\bm{Z}_0 \bm{\gamma}_0$ and $\bm{Z}_1 \bm{\gamma}_1$ representing the design matrices and coefficients of the random intercepts and random slopes respectively. Explicitly, the coefficients are formulated as $\gamma_{0}=\left(\bm{\gamma}_{01}, \ldots, \bm{\gamma}_{0 i}, \ldots, \bm{\gamma}_{0 m}\right)'$ and $\gamma_{1}=\left(\bm{\gamma}_{11}, \ldots, \bm{\gamma}_{1 i}, \ldots, \bm{\gamma}_{1 m}\right)'$, whereas the design matrices are expressed as
$$
\bm{Z}_0 =
\left(
\begin{matrix}
\bm{1}_1 &  &  &  & \bm{0} \\ 
 & \ddots &  &  &  \\ 
 &  & \bm{1}_i &  &  \\ 
 &  &  & \ddots &  \\ 
 &  &  &  & \bm{1}_m
\end{matrix} 
\right)
\qquad
\bm{Z}_1 =
\left(
\begin{matrix}
\bm{x}_1 &  &  &  & \bm{0} \\ 
 & \ddots &  &  &  \\ 
 &  & \bm{x}_i &  &  \\ 
 &  &  & \ddots &  \\ 
 &  &  &  & \bm{x}_m
\end{matrix} 
\right).
$$
More details and technicalities regarding mixed effects in additive models can be found in the respective literature.\\


Extensions of additive models to non-normal responses are consequently called \textit{\acp{GAM}}, which were first introducd by \cite{hastie1986}. If additionally random effects are included, they are called \textit{\acp{GAMM}}.\\

Thus far, models with main effects and conceivably random effects have been introduced. Accordingly, these types of effects can likewise be combined with covariate interactions and/or spatial effects. Such models can be described in a unified framework and are titled as (possibly \textit{Generalized}) \textit{\acp{STAR}},
$$
y=f_{1}\left(\nu_{1}\right)+\ldots+f_{q}\left(\nu_{q}\right)+\beta_{0}+\beta_{1} x_{1}+\ldots+\beta_{k} x_{k}+\varepsilon.
$$
The covariates $\nu_1, \ldots, \nu_q$ can be one- or multidimensional and the functions can be of different structure determining the type of effect.





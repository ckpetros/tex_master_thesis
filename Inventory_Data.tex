% !TEX root = Master.tex

In spring 2018, a sample-based forest inventory was carried out in Gartow. 942 sampling locations are defined which are spread over the forest based on a stratified sampling approach. This accounts for the past observed variation within the regions. Compartments of stratum 1 and 2 are sampled with a dense sampling grid, while 3 and 4 have a wider sampling grid (see Table \ref{tab:Sample_Variation} \& Figure \ref{fig:Gartow Map}).\\

At each sampling location (so called plots) several attributes of the trees within a certain circular area are measured. The parameters of primary interest in this study are the diameter, species and height. The diameter is measured at breast height (around 1.3 meters) with a measuring tape. Subsequently, the height is measured with varying, but established methodologies. Unlike the diameter, not every tree height is collected. In each plot, three main species trees (less if there are fewer trees) are measured. To cover the total range of values, a small, a medium sized and a large one is gauged. Additionally, one tree of every other species is measured to cover the variety of species. Table \ref{tab:Measured Trees} provides an overview of total measured trees.

\begin{table}[H]
\setlength\arrayrulewidth{1pt}  
\centering
\begin{tabular}{|c| c|}
\hline 
\rowcolor{Gray}
\textbf{Stratum} & \textbf{\# Measured Trees} \\ 
\hline 
1 & 1287 \\ 
\hline 
2 & 3434 \\ 
\hline 
3 & 2734 \\ 
\hline 
4.1 & 616 \\ 
\hline 
4.2 & 792 \\ 	
\hline 
G & 523 \\ 
\hline 
GL & 619 \\ 
\hline 
\rowcolor{SeaBlue}
Total & 10005 \\ 
\hline 
\end{tabular} 
\caption{Overview of number of measured trees for height and diameter per Stratum}
\label{tab:Measured Trees}
\end{table}


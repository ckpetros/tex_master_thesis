% !TEX root = Master.tex

To predict the diameter of a tree using regression models is straight forward. We chose the later described regression models based on following criteria. \\

\renewcommand{\labelenumi}{\arabic{enumi}.}
\begin{enumerate}

\item \textit{Interpretability} is required so experts can understand the results to continue to work with the proposed models. Inference on all covariates must be feasible, as large standard errors (i.e. high uncertainty in the predicted diameter) causes the model to be unusable.

\item \textit{Generalization} (i.e. the goodness of fit to new data) must be high to assure sufficient prediction of the diameter
of trees detected by LiDAR. While it is likely that regression splines might result in an overall better fit to the data,
relying too much on the sampling data might cause substantial over-fitting. As described in Section \ref{Challenges},
over-fitting must be prevented due to the unreliability in the data. Regularization methods such as ridge
regression could be applied. Yet, based on the results in the applied models it was not further investigated.

\item \textit{Complexity} should be kept moderate to again make the models more accessible. The model is only used in
this study to predict the diameter. Using Generalized Additive Models for Location Scale and Shape could be an
interesting addition to this study to further investigate the uncertainty at different in the estimates but is potentially too
complex for the task of a simple prediction.

\end{enumerate}

Based on those criteria, a log-linear model (based on the observations in Section \ref{Overview variables}) and generalized linear models are
compared.
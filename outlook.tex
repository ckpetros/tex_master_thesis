% !TEX root = Master.tex



With \acp{DBN}, causal effects of various sets of data can be estimated and the challenges of poor data quality or insufficient data, like discussed in Section \ref{ssec:individual_patterns}, can be overcome.\footnote{\cite{bn_advantages} gave an interesting read about this topic.} Directed acyclic graphs of other data subsets can be produced to assess conditional probabilities between sales of distinct articles or other objects. For example, one might delimit the data such that only running shoes across multiple clusters (e.g. business units) can be compared, which would be arguably legitimate. Although \acp{DBN} capture causal effects, which can be interpreted in terms of product cannibalization, one drawback is that sales transferability\footnote{i.e. The actual quantity flow of unit sales of one article towards another across successive time-points.} cannot be measured (which also constitutes an interesting research question). Another point of interest would be the involvement of explanatory variables, such as the associated weekly price tags or promotion intensities of the articles.


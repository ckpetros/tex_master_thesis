% !TEX root = Master.tex
\textit{\acp{LMM}} are powerful tools when dealing with clustered data or data with a longitudinal structure (repeated measurements of individuals).
As in the classical \ac{LM}, there are population-specific effects, namely the parameter vector of \textit{fixed effects} $\boldsymbol{\beta}$, as well as the cluster- or individual-specific effects of such models called \textit{random effects} \citep{fahrmeir2003regression}. In the following, we will refer to our clusters or individuals as "groups" for briefness.\ Mathematically speaking, the linear predictor $\eta_{ij}= \mathbf{x}'_{ij} \mathbf{\beta} $ is extended to
\begin{equation}
\eta_{ij} = \bm{{x'}}_{ij} \bm{\beta} + \bm{u'}_{ij}\bm{\gamma}_i, \quad j=1, \ldots, m, \quad i=1, \ldots, n_i, 
\label{eq:linear_mixed_predictor}
\end{equation}
where
\begin{itemize}
\item $i$ is the number of groups
\item $j$ is the number of observations per group
\item $\bm{\beta}$ is the vector of fixed effects
\item $\bm{\gamma}_i$ is the vector of random effects
\item $\bm{x'}_{ij}$ is the vector of covariates and
\item $\bm{u'}_{ij}$ is a subvector of $\bm{x'}_{ij}$.
\end{itemize}


$\bm{x'}_{ij} = (1, x_{ij1}, \ldots, x_{ijk}) $ and $ \bm{u'}_{ij} = (1, u_{ij1}, \ldots, u_{ijk}) $ are therefore the design vectors and $\varepsilon_{ij}$ are the error terms of the \textit{measurement model}

\begin{equation}
y_{ij} =  \bm{x'}_{ij} \bm{\beta} + \bm{u'}_{ij} \bm{\gamma}_i + \epsilon_{ij}, \quad \varepsilon_{i j} \overset{i.i.d.} \sim N\left(0, \sigma^{2}\right) 
\end{equation}
or in matrix notation

\begin{equation}
\bm{y}_{i}=\bm{X}_{i} \bm{\beta} + \bm{U}_{i} \bm{\gamma}_{i} + \bm{\varepsilon}_{i}
\end{equation}
for group $ i=1, \ldots, m $ with $ E(\bm{\varepsilon}_{i}) = \bm{0}$. \\

Similar to \acp{GLM}, \textit{\acp{GLMM}} relate the linear mixed predictor \ref{eq:linear_mixed_predictor} to the conditional mean 
$ \mu_{ij} = E(y_{ij} | \bm{\gamma}_{i}) $
via a suitable response function $h$, such that
$ \mu_{ij} = h(\eta_{ij}) $ and thus the conditional density of $y_{ij}$ belongs to the exponential family.
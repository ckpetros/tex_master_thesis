% !TEX root = Master.tex

To compensate some of the drawbacks of linear correlation, we can take advantage of correlation measures based on the ranks of data. \textit{Rank correlation coefficients}, like the ones presented below, are always defined and obey to the invariance principal. This means that these coefficients only depend on the underlying copula and they can thereof be directly derived.\\

\textbf{Spearman's Rho}\\
Consider two \acp{RV} $X_1$ and $X_2$ with continuous \acp{CDF} $F_1$ and $F_2$, then the \textit{Spearman's rho correlation coefficient} is simply the linear correlation between the \acp{CDF}
\begin{align}
\rho_{S}=\rho\left(F_{1}\left(X_{1}\right), F_{2}\left(X_{2}\right)\right).
\end{align}
The reason being is that by applying the \ac{CDF} to data, naturally a multiple of the ranks of the data are obtained, which essentially is equivalent to
\begin{equation}
\rho_S = \rho ( Ran(X_1), Ran(X_2) )
\end{equation}
Due to the invariance principle, we also obtain Spearman's rho directly from the unique copula via
\begin{equation}
\rho_{\mathrm{S}}=12 \int_{0}^{1} \int_{0}^{1} C\left(u_{1}, u_{2}\right) \mathrm{d} u_{1} \mathrm{d} u_{2}-3.
\end{equation}


\textbf{Kendall's Tau}\\
Let $X_1 \sim F_1$ and $X_2 \sim F_2$ be two \ac{RV} and let $(\tilde{X}_{1}, \tilde{X}_{2})$ be an independent copy\footnote{An independent copy $\tilde{X}$ of a RV $X$ is a RV that inherits from the same distribution as $X$ and is independent of $X$.} of $({X}_{1}, {X}_{2})$. Then \textit{Kendall's tau} is defined by 
\begin{equation}
\begin{aligned}
\rho_{\tau} &={E}\left[\operatorname{sign}\left(\left(X_{1}-X_{1}^{\prime}\right)\left(X_{2}-X_{2}^{\prime}\right)\right)\right] \\
&={P}\left(\left(X_{1}-X_{1}^{\prime}\right)\left(X_{2}-X_{2}^{\prime}\right)>0\right)-{P}\left(\left(X_{1}-X_{1}^{\prime}\right)\left(X_{2}-X_{2}^{\prime}\right)<0\right).
\end{aligned}
\end{equation}
Similarly to Spearman's rho, using the invariance principal, we can directly derive Kendall's tau from the unique copula by
\begin{equation}
\rho_{\tau}\left(X_{1}, X_{2}\right)=4 \int_{0}^{1} \int_{0}^{1} C\left(u_{1}, u_{2}\right) d C\left(u_{1}, u_{2}\right)-1.
\label{eq:kendall_to_copula}
\end{equation}


Both $\rho_S, \rho_{\tau} \in [-1,1]$ and any value within this interval is attainable for an arbitrary copula class in contrast to the Pearson coefficient. 
If any of these rank correlations is $-1$ (or $1$), we are in the countermonotonic (or comonotonic) case. If $\rho_S$ (or $\rho_{\tau}$) $=0$, this does not necessarily imply independence between $X_1$ and $X_2$, although the opposite direction holds.
Furthermore, they are not limited to be invariant just under linear transformations . 











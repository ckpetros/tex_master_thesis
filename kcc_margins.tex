% !TEX root = Master.tex

\autoref{fig:kcc_pair_scatterplots} (see Section \ref{ssec:grouped_patterns}) is hinting that the marginal distributions come with a noticeable skewness, which are best to take into account. The logarithmic scale is the preferred transformation due to variance stabilization of the margins. Among a pool of possible parametric distributions, we pick an appropriate one for each margin. Several distributions would theoretically be justifying the shape of our data, e.g. Weibull, Gamma, Box-Cox-Cole-Green or Dagum distribution. After screening those parametric distributions, we find that the exponentially modified Gaussian distribution (or \textit{ex-Gaussian} distribution which is used from here on) fits all three margins fairly well \citep{grushka1972characterization}.\footnote{Another appropriate distribution would be the Dagum distribution \citep{dagum1975model}, however interpretability of the parameters is difficult to comprehend.} Additionally, the ex-Gaussian distribution has readily understandable parameters, including a desired skewness parameter.
%Although the distribution assumptions are aligned with the empirical distributions of the data, there are still some expected outliers. This is quite visible in Figures \ref{fig:kcc_2_margin} to \ref{fig:kcc_8_margin}, which depict the marginal empirical distributions along with the theoretical assumptions, especially when looking at the Quantile-Quantile Plots (right figures).
Before entering the dependence structures between the three \ac{KCC} pairs, the marginal distribution of each cluster is analyzed individually in the next three subsections. First, the ex-Gaussian distribution is fitted to the margins and afterwards covariate effects are included within the context of \ac{GAMLSS} to obtain flexible estimations on the distribution parameters. 
%\footnote{Maximum likelihood estimation of a simple \ac{GAMLSS} model (see \cite{rigby2005generalized}).} 




%For the approaches in the below Section \ref{ssec:kcc_correlations} requiring parametric distribution assumptions for the margins, we will consider the Dagum distribution with parameters specified as in the captions of the above three figures, which represent estimated maximum likelihood values (that is, if they do not have to be estimated over again within a model like e.g. \ac{GAMLSS}).



